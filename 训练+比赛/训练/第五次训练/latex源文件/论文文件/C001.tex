\documentclass{whutmod}
\usepackage{metalogo}
\usepackage{setspace}
\usepackage{subfigure}
\usepackage{caption}
\hypersetup{
colorlinks=true,
linkcolor=black
}
\usepackage{amsmath} 
\usepackage{amssymb}
\usepackage{float}
\usepackage{booktabs}
\usepackage{graphicx}


\team{1}	% 组号
\membera{阮滨}
\joba{编程}
\memberb{王宇}
\jobb{建模}
\memberc{王家柯}
\jobc{写作}

\title{基于指派问题模型的指纹识别与匹配}
\tihao{2} % 题号

\begin{document}
\begin{spacing}{1.2}
	

\maketitle

\begin{abstract}\setlength{\parskip}{0.3\baselineskip}
本文首先通过图像分割、图像增强等方法对指纹图像进行了预处理,并结合全局特征点和细节特征点构建了具有旋转、平移不变性的指纹特征,
之后通过特征编码将其以较少的字节数进行存储。此外,本文运用指标归一化后的欧氏距离对指纹特征点相似程度进行表达,进而以最大化最佳
的特征点匹配方案下的指纹相似程度作为目标,将指纹匹配问题转化为指派问题进行求解。最后,运用方向场信息,对指纹的纹型进行分类。

由于附件中的指纹图像存在模糊、损坏等情况,故需要对指纹图像进行预处理。本文首先采用了\textbf{基于梯度向量模的图像分割算法},并通过邻域分析消除了独立
背景块。然后通过\textbf{Gabor滤波}对图像进行了增强,突出了指纹的脊线特征。考虑到全局阈值二值化方法对图像信息的破坏,本文采用了\textbf{结合方向信息的自适
应动态阈值二值化方法}。最后采用了\textbf{改进的OPTA算法}对图像进行细化,使指纹的特征点信息更为明显。

针对问题一需要获得指纹的特征并进行编码。考虑到指纹图像可能存在的旋转、平移以及失真,本文通过\textbf{结合全局特征点和细节特征点,
构建具有旋转、平移不变性的极坐标系},增强特征的可靠性。此外,由于手指在按压的过程中,外围的指纹因皮肤的弹性受到拉伸,使得
特征信息产生误差,故仅在中心点的邻域内选取特征点,进而本文选取了包括\textbf{极坐标、与中心点的方向场角度差、特征点种类等五个指标}。
最后将其以\textbf{特征编码}的方式进行存储,并通过计算发现,在题目的编码字节的限制下,每个指纹最多可存储22个特征点。

针对问题二需要在上述编码的基础上衡量指纹的相似度。本文首先对问题一中每个细节特征点的五个指标进行了\textbf{归一化处理},并通过\textbf{欧式距
离}计算了特征点之间的相似程度。同时,考虑到相同指纹中各特征点的唯一对应关系,可以认为必须在两个指纹的特征点在达到整体最佳配
对的前提下才可以合理的计算其相似程度。因此,本文将指纹匹配问题转化为\textbf{指派问题}进行分析,并以具有惩罚图像不完整性性质的相似性
得分计算方法作为最大化目标,构建起\textbf{具有一定误差包容空间的整数规划模型},进而选取了4组算例对模型进行验证,通过LINGO得到相似性
得分分别为\textbf{0.89,0.67,0.52,0.55},基本符合主观验证的结果。

针对问题三需要对附件中的指纹图像进行对比分类。本文首先根据前文所求的\textbf{方向场信息},运用\textbf{Poincare Index算法}得到指纹的全局关键点,
并通过关键点中核心点和三角点的个数对指纹的纹型进行初步判断,再通过核心点和三角点连线上的方向角均值最终确定指纹纹型。最后
通过与主观分析的结果进行对比,发现算法对于纹型的识别成功率可达\textbf{81.25\%}。

本文的优点为:1.将复杂的指纹识别问题转化为经典的指派问题,极大简化了问题的复杂程度;2.通过0-1变量的非线性项线性化,提高了模型
的求解速度;3.对指纹图像进行了充分的预处理,保证了后续分析的准确性。


	\keywords{
		指纹识别 \quad
		指派问题\quad
		方向场 \quad
		模式识别
		 
		

	}
\end{abstract}

\tableofcontents
\newpage

\section{问题重述}

\subsection{问题背景}
随着城市化进程逐步发展和经济的日益增长,城市的人均拥有机动车量越来越高,这就导致了交通日益拥堵
和环境的恶化。因此提倡市民出行使用公交车是解决问题的途径之一,提高公交车的服务水平是吸引市民
采取公共交通的关键,因此需要对公交系统运营效率进行优化,减少市民等待公交车的时间和高峰期的
拥挤程度。本文结合问题尝试给出一个弹性的公交车调度方案。


\subsection{问题重述}

问题一:结合文献以及相关数据,给出一条公交线路的“高峰”和“平峰”的定义,并利用数学语言说明给出定义的合理性及有效性。


问题二:结合问题一给出的高峰和平峰的定义,给定一条数据。给出“转换期”的调度方案模型,并提出判断方案可行的指标,
进一步去评价判断参数对于此模型的稳健性和灵敏性。

问题三:建立能够预测高峰和平峰的数学模型。

问题四:结合收集到的实际数据说明模型的可靠性。


\section{模型的假设}

\begin{itemize}
	\item 站间公交车匀速行驶,不考虑交叉口、信号灯以及交通路阻的影响;
	\item 假设本文讨论的纹型均为数据库可查询的已知纹型;
	\item 假设特征点只考虑端点和交叉点;
	\item 假设特征点之间是相互独立的。
\end{itemize}

\section{符号说明}
\begin{center}
	\begin{tabular}{cc}
		\\\toprule[1.5pt]
		\makebox[0.3\textwidth][c]{符号}	&  \makebox[0.4\textwidth][c]{意义} \\ \hline
		$\partial(i,j)$	    & 梯度向量模 \\ 
		$\theta (m,n)$	    & 块方向  \\ 
		$I$           & 特征点种类  \\ 
		$P_0(x_c,y_c)$   & 极点坐标 \\ 
		$d_{ij}$ & 两点之间的欧式距离 \\ 
		$M_{MO}$   &  匹配成功点的个数 \\ 
		\bottomrule[1.5pt]
	\end{tabular}
\end{center}
注:其他未列出的符号以第一次出现时的解释为准

\clearpage



\section{问题一的模型建立与求解}


\subsection{问题分析}
记在时间区间$(0,t](t>0)$内到达车站的通勤者为$N(t)$,在时间区间$(t_1,t_2](t_2>t_1)$内到达n个
通勤者的概率为$P_n(t_1,t_2)=P{N(t_2)-N(t_1)=n}(n\ge 0)$.

由于通勤者乘车排队满足泊松流的三个条件:

1、无后效性。在不相互重叠的时间区域内,通勤者到达数是相互独立的。

2、平稳性。在时间区域$[a,a+t)$内有n个通勤者到达车站的概率与$P_n(a,a+t)$与$a$无关,仅仅与t和n有关。

3、普通性。对于充分小的$\Delta t$,在时间区间$[t,t+\Delta t)$内有两个或两个以上通勤者概率极小,
即有$P_1[t,t+\Delta t)=\lambda \Delta t +o(\Delta t)$,$\sum_{n=2}^{\infty}P_1(t,t+\Delta t)=o(\Delta t)$,
其中$\lambda >0$是常数,$o(\Delta t)$是$\Delta t$的高阶无穷小。

当输入过程为泊松流时,在长为t的时间区间内到达n个通勤者的概率遵从泊松分布,即公式(\ref{1}):
\begin{equation}
	\label{1}
	\begin{split}
		P_n(a,a+t)=P_n(0,t)=\frac{(\lambda t)^n}{n!}e^{-\lambda t} \quad (t>0,n=0,1,2,...),
	\end{split}
\end{equation}

而在单位长的时间区间内到达的平均顾客数为$\lambda$,即$E(N(a+1)-N(a))=\lambda.$


\begin{figure}[H]

\subsection{建立基于指纹局域结构特征的编码模型}

\subsubsection{基于曲率场的伪中心点选取}
指纹的曲率场表示的是沿着固定长度的弧度,脊线方向的变换量,一般通过跟踪细化后的指纹极限获得。
在指纹的某些区域(例如核心点区域)曲率值的变换遍布整个指纹图像。Yager et al.利用三次薄板样条
法表示每条细化后的脊线,从而计算出某些像素点的切线方向上的变换率,最后使用高斯滤波器来获取整张
指纹图像的脊线曲率图。

薄板样条法:从细化后的图片中,获取离散的脊线。每条脊线用一个坐标集合表示,$(x_i,y_i)$和
$(x_{i+1},y_{i+1})$表示离散脊线上相邻的两个像素的坐标位置。根据曲率的定义,第$i$个像素点
的曲率值的计算如


\subsubsection{特征点的选取}
(1)指纹特征端点的提取

指纹特征端点是指纹纹线的两头末端,即终点和起点。在纹线像素模型的九点图中,如果中心点黑色方块
是端点那么该黑方块的上下左右点显然只有两个,由于白点块的像素灰度值为255,那么端点周围8个点
两两相邻的两点差的绝对值之和就是$2\times 255$,满足以此标准即确定中心黑色方块为端点。下图为
在八邻域的所有状态中满足特征端点特征条件的八种情况$^{[6]}$。
\begin{figure}[H]
	\centering
	\includegraphics[width=.4\textwidth]{端点.png}
	\caption{端点模板}
	\label{断电模板}
\end{figure}

提取方法为扫描某个点,如果周围8个点两两相邻的两个点的差的绝对值和为$2\times 255$
,则为端点。

(2)指纹特征分叉点的提取

对于指纹特征的选取我们只要选取指纹的特征分叉点是两条纹线交会为一条纹线的交汇点。在纹线分叉点
像素模型的九点图中,如果中心黑色方块点是分叉点,那么去掉分叉点后,纹线剩下三个黑色方块,每个
黑色方块上下相邻的白色方块有两个,由于白色方块的像素值为255,那么分叉点周围8个两两相邻的两个点
差的绝对值之和就是$6\times 255$,满足此标准即确定中心黑色方块为分叉点。下图为在八邻域的所有状
态中满足特征端点特征条件的八种情况。

\begin{figure}[H]
	\centering
	\includegraphics[width=.4\textwidth]{分叉点模板.png}
	\caption{分叉点模板}
	\label{分叉点模板}
\end{figure}


提取方法为,扫描某个点,如果周围如果周围8个点两两相邻的两个点的差的绝对值和为$6\times 255$
,则为分叉点。以下为特征点提取后的图片,其中端点用绿色圈出,分叉点用红色圈出。




\subsubsection{基于局部结构和细节特征点的指纹编码}
1.细节特征点的筛选

考虑到编码的字节限制,故有必要对细节特征点进行筛选。另外,由于皮肤的弹性,越靠近边缘的指纹
越容易受到非线性扭曲的影响,因此本文仅选取临近中心点的部分细节特征点。

本文设定阈值$R$,并认为细节特征点与中心点$P_0(x_0,y_0)$的距离小于此值时才纳入编码。进而通过
遍历所有细节特征点,结合上述的特征计算方法,可以得到相应的特征集合$S$,可表示如下:
\begin{equation}
	\label{2.6}
	\begin{split}
		 S={(x_i,y_i,\theta_{fi},I_i,n_i)|\sqrt{(x_i-x_{0i})^2+(y_i-y_{0i})^2}\leq R,I_i\in{a,b,c} ,i=1,2,...,N}
	\end{split}
\end{equation}

其中$I$表示细节特征点的种类,若$I=0$,表示该特征点为中心点,$I=1$则表示该特征点为端点,$I=2$
则说明该特征点为分叉点,$n$为中心点与特征点连线穿过的脊线数,$\theta_f$为特征点所在区域的块
方向场角度。

需要注意的是,为保证满足字节限制的要求,若在半径为$R$的领域内存在过多的特征点,本文仅选取距
离中心点最近的30个特征点。

2.构建极坐标系

将中心点$P_0(x_c,y_c)$作为极点,其所在方向场的角度为$\theta_c$,沿此角度作射线$oP_0$作为极轴,
构建具有旋转平移不变性的极坐标系。


3.坐标系转换

为将上述的特征集合映射至构建的极坐标系中,需计算出特征点坐标与中心点的坐标以及两点所在方向场的
方向差。

\begin{equation}
	\label{2.6}
	\begin{split}
		\left\{
			\begin{aligned}
			\rho & =  \sqrt{(x-x_0)^2+(y-y_0)^2} \\
			\theta& = arctan\frac{y-y_0}{x-x_0} -\theta_c \\
			\ \Delta\theta & = \theta_f-\theta_{0f}  \\
			\end{aligned}
			\right.
	\end{split}
\end{equation}


故特征集合修改为:
$$ S=\{(\rho_i,\theta_i,\Delta\theta_i,I_i,n_i)|\rho_i \leq R,I_i\in\{0,1,2,3\},i=1,2,...,N \}$$
其中$\theta,n,I$的含义与上文相同。

4.特征编码
本文采取特征向量的方式进行编码,即使得一个特征点的各特征作为一列内的元素。同一个指纹中的$N$
个特征点便对应有$N+1$列向量,其中第一列为中心点的坐标,最后整合存储。








\section{问题二模型的建立与求解}

\subsection{问题分析}

根据上文确定的特征选取方法,在此主要解决如何基于向量编码所提供的信息进行指纹的匹配。文本首先拟采用欧式距离衡量任意两
个指纹之间的相似程度,并构建起指纹相似距离矩阵。结合相同指纹特征点一一对应的性质,易将此问题转化为指派问题进行分析。

在模型的建立过程中,还需要考虑到相似性的衡量方法以及较小误差的容错空间。对于前者,本文采用了Ravi基于惩罚图像不完
整性的匹配分数计算方法$^{[7]}$;而对于后者,本文拟通过0-1变量的约束使得唯有两指纹差异大于某一阈值时
才认定匹配失败。

\subsection{指纹编码结果}

以附件中的图一为例,其指纹的编码结果为:44.283 -1.510 0.748 1 4
46.872 -1.365 0.748 1 5
51.740 -1.236 0.805 1 6
19.026 -1.901 -0.459 1 0
13.928 -1.586 -0.286 1 1
28.792 -0.737 -0.435 1 2
65.521 -0.661 -0.616 1 6
71.868 0.818 -1.634 1 5
81.988 0.664 -2.331 1 6
34.132 -1.865 -0.358 2 0
96.773 -0.143 -0.692 2 12
39.051 -0.615 -0.499 2 3
51.225 0.513 0.760 2 4。

剩余的指纹编码结果见附录A。

\subsection{基于指派问题的指纹异同分析模型}

考虑到理想的完全匹配状态下,两个指纹上的细节特征点应一一对应。但实际上
,指纹采集质量或是指纹来源不同都有可能使得指纹的细节特征对应关系被破坏$^{[8]}$。

若在计算相似程度时,将两个指纹上的细节特征点错配、重复配,则会导致相似程度的计算出现较大误差$^{[9]}$。
为此,本文通过将指纹的细节匹配问题转化为指派问题进行分析,使得在最佳匹配关系的基础上判断指纹相似程度。故首先需要计算任意两个指纹之间的相似性,并汇总为相似性矩阵。

两指纹之间的相似性通过如下方式定义:

Step1:将各特征点在上述相对极坐标系中的坐标$(\theta_i,\rho_i$),
穿过脊线的条数$n_i$,特征点与中心点的方向场角度差$\Delta\theta_i$,特征点的种类$I_i$都进行归一化。以特征点与中心点穿过脊线的条数为例:
$$ n_k'=\frac{n_k-\min(n_i)}{\max(n_i)-\min(n_i)}$$

Step2:运用欧氏距离计算任意两个指纹之间的差异性。

$$d_{ij}=\sqrt{(\theta_i-\theta_j)^2+(\Delta\theta_i-\Delta\theta_j)^2+(\rho_i-\rho_j)^2+(I_i-I_j)^2+(n_i-n_j)^2}$$

Step3:构建相似性矩阵



进而构建指派问题的整数规划模型,首先选取Ravi提出的匹配分数计算方法作为目标函数:
$$\max Score=\frac{M_{MO}}{\max(N_1,N_2)}$$

其中$M_{MO}$为匹配成功的点的个数,$N_1$录入细节点的总数目,$N_2$模板细节点的总数目。这种匹配分数的计算方法在一定程度上惩罚了不完整指纹图像匹配以及重叠部分较少的情况。

根据特征点的一一对应原则建立相应的约束:

\begin{equation}
	\label{2.6}
	\begin{split}
     \begin{cases}
		 \sum_{i=1}^{N_1}x_{ij}\le1 \quad ,j=1,2,...,N_2;\\
		 \sum_{j=1}^{N_1}x_{ij}\le1 \quad ,i=1,2,...,N_1.
	 \end{cases}
	\end{split}
\end{equation}

实际匹配过程中允许存在一定误差,故两点之间相差距离$d_{ij}$超过阈值$L$时才认定匹配失败:

\begin{equation}
	\label{2.6}
	\begin{split}
	 (1-Z_{ij})\ge X_{ij}d_{ij}-L  \quad i=1,2,...,N_1,j=1,2,...,N_2.
	\end{split}
\end{equation}

通过0-1变量的约束得到成功匹配的特征点总数:
\begin{equation}
	\label{2.7}
	\begin{split}
		M_{MO}=\sum_{i=1}^{N_1}\sum_{j=1}^{N_1}Z_{ij}X_{ij}.
	\end{split}
\end{equation}

为提高求解速度,本文将上式线性化,即通过(\ref{2.8})代替(\ref{2.7}):




\begin{equation}
	\label{2.8}
	\begin{split}
		\begin{cases}
			M_{MO}=\sum_{i=1}^{N_1}\sum_{j=1}^{N_1}p_{ij}\\
			p_{ij}\le X_{ij},\quad i=1,2,...,N_1,j=1,2,...,N_2;\\
			p_{ij}\le Z_{ij},\quad i=1,2,...,N_1,j=1,2,...,N_2;\\
			Z_{ij}+ x_{ij}\le p_{ij},\quad i=1,2,...,N_1,j=1,2,...,N_2.
		\end{cases}
	\end{split}
\end{equation}




综上,将模型汇总可得到:

\begin{equation}
	\label{3.2}
	\begin{split}
		&\max Score=\frac{M_{MO}}{\max(N_1,N_2)}\\
		&\begin{cases}
		\sum_{i=1}^{N_1}x_{ij}\le1 \quad ,&j=1,2,...,N_2;\\
		 \sum_{j=1}^{N_1}x_{ij}\le1 \quad ,&i=1,2,...,N_1;\\
		 (1-Z_{ij})\ge X_{ij}d_{ij}-L  \quad &i=1,2,...,N_1,j=1,2,...,N_2;\\
		 M_{MO}=\sum_{i=1}^{N_1}\sum_{j=1}^{N_1}p_{ij}; &\\
		 p_{ij}\le X_{ij}  \quad \quad &i=1,2,...,N_1,j=1,2,...,N_2;\\
		 p_{ij}\le Z_{ij}  \quad \quad &i=1,2,...,N_1,j=1,2,...,N_2;\\
		 Z_{ij}+ x_{ij}\le p_{ij}  \quad  &i=1,2,...,N_1,j=1,2,...,N_2;\\
		 Z_{ij}, x_{ij}, p_{ij} = 0 or 1
		\end{cases}
	\end{split}
\end{equation}

\subsection{指派问题的模型求解}

本文选取了附件中的01.tif与02.tif、05.tif与06.tif、07.tif与08.tif,以及10.tif和11.tif作为4组算例,
并基于LINGO编程实现了上述模型。

\begin{figure}[H]
	\centering
	\includegraphics[width=.8\textwidth]{指纹对比.png}
	\caption{指纹对比示意图}
	\label{指纹对比}
\end{figure}

% Table generated by Excel2LaTeX from sheet 'Sheet2'
\begin{table}[H]
	\centering
	\setlength{\abovecaptionskip}{0pt}
	\caption{0.7阈值下的对比得分}
	  \begin{tabular}{ccccc}
		\\\toprule[1.5pt]
	   指纹5和指纹6 & 指纹1和指纹2 & 指纹7和指纹8 & 指纹10和指纹11 \\\hline
	   0.789  & 0.556  & 0.552  & 0.478  \\
	  \bottomrule[1.5pt]
	  \end{tabular}%
	\label{tab:addlabel}%
\end{table}%
  
~\\
结合主观分析,发现对于较为相似的图7与图8,模型得分较高;而对于差异较大的图10和图11,
模型得分较低。与此同时,我们也注意到都为螺旋形的图7和图8以及都为环形的图1和图2显然
具有一定相似性,但在得分上未能有明显区别,故接下来进行结果分析,对阈值参数的选取进行讨论。

\subsection{结果分析}
结合灵敏度分析的基本思想,本文以0.1为步长,初选阈值$L=0.7$的领域内选取了其它5个阈值,
再次对上述的四个算例进行求解。结果如下:
% Table generated by Excel2LaTeX from sheet 'Sheet1'
\begin{table}[H]
	\centering
	\setlength{\abovecaptionskip}{0pt}
	\caption{不同阈值对比分析图}
	\begin{tabular}{ccccccc}
		\\\toprule[1.5pt]
	   & 0.5 & 0.6 & 0.7 & 0.8 & 0.9 & 1 \\\hline
	   指纹5和指纹6 & 0.632  & 0.789  & 0.789  & 0.895  & 0.895  & 0.895  \\
	   指纹1和指纹2 & 0.389  & 0.444  & 0.556  & 0.667  & 0.722  & 0.722  \\
	   指纹7和指纹8 & 0.345  & 0.438  & 0.552  & 0.552  & 0.586  & 0.655  \\
	   指纹10和指纹11 & 0.364  & 0.409  & 0.478  & 0.545  & 0.545  & 0.545  \\
	  \bottomrule[1.5pt]
	  \end{tabular}%
	\label{tab:addlabel}%
\end{table}%
  
可以发现当阈值降低时,模型整体分数降低;反之升高。而在模型的设定中,距离超过阈值时,
则认为匹配失败,故与观察到的现象相符。

此外,对于过高或过低的阈值,都出现了指纹得分差异不明显的情况,使得指纹之间的相似性衡
量准确性降低,因此在给定算例的前提下,本文给出如下公式调整阈值$L$:

$$L=argmax (\sigma (L,K)).$$

其中$K$为选定的算例集合。

在上述算例以及讨论的阈值$L$中,$L=0.8$为最优参数。


\clearpage

\section{问题三模型建立与求解}


\subsection{问题分析}
针对附件中给的16张指纹图片,通过对指纹图像的处理,得到指纹分类所需要的信息,建立一个基于中心点
和三角点这些特征点的指纹分类算法。



\subsection{基于特征点的指纹分类模型}

\subsubsection{三角点和中心点的检测}

由于中心点周围纹线呈半圆趋势,三角点周围的纹线由三个部分组成,而每个部分都呈现双曲线形状
这些固有的几何特点,在本文中通过计算像素点周围的Poincare值来得到三角点和中心点,如Poincare
值为$\frac{1}{2}$,则此处得到一个core点,若Poincare值为$-\frac{1}{2}$,则此处得到一个delta点。

\begin{figure}[H]
	\centering
	\includegraphics[width=.5\textwidth]{端点示意.png}
	\caption{core和delta端点示意图}
	\label{端点}
\end{figure}

(1)core点的检测:

设$(i,j)$为当前子图像的中心像素,$(i-1,j)$、$(i-1,j+1)$、$(i,j+1)$分别为其相邻的子图像的
中心,假设$O(i,j)$为$(i,j)$处的方向值,令$X_x(.)$和$X_y(.)$分别为具有N个中心像素的闭合
数字曲线的x与y坐标值,此处N=4。现在顺次求每个$(i,j)$处的Poincare值$^{[10]}$,即以逆时针方向求调整
后的这个N子图像中心的方向信息差值的累积和:



\begin{equation}
	\label{3.3}
	\begin{split}
       Poincare(i,j)=\frac{1}{2\pi}\sum_{i=0}^N\Delta(k)
	\end{split}
\end{equation}

其中

\begin{equation}
	\label{3.3}
	\begin{split}
		\Delta(k)=\begin{cases}
			\partial(k)& \lvert \partial(k)\rvert < \frac{\pi}{2}\\
			\pi+\partial(k)& \lvert \partial(k)\rvert \le -\frac{\pi}{2}\\
			\pi-\partial(k)&else\\
		\end{cases}
	\end{split}
\end{equation}


\begin{equation}
	\label{3.4}
	\begin{split}
		&\partial(k)=O(\Psi_x(i')\Psi_y(i'))-O(\Psi_x(i)\Psi_y(i)).\\
		&i'=(i+1)modN.
	\end{split}
\end{equation}

若$Poincare(i,j)$等于$\frac{1}{2}$,则此点为core点。

(2)delta点的检测:

假设$(i,j)$为当前子图像中心像素,$(i-1,j)-1$、$(i-1,j+1)$、$(i+1,j-1)$、$(i+1,j+1)$
分别为与其相邻子图像的中心像素,同上沿着此四点求取各$(i,j)$处的Poincare$(i,j)$等于$-\frac{1}{2}$
则此点检测到一个delta点。

得到core点和delta点的数目后,即开始进行分类,这里采取有反馈的指纹分类方法,首先检测core点和delta点
的对视,若core点和delta点对数大于2,则对图像进行一次平滑,然后再分类(因为几乎没有core点或者delta多于2个的模型)
,否则直接按core点和delta点的个数与其相对位置进行分类。

\begin{figure}[H]
	\centering
	\includegraphics[width=.5\textwidth]{特征对比.png}
	\caption{特征点示意图}
	\label{纹型判断}
\end{figure}


\subsubsection{分类原则}

\begin{figure}[H]
	\centering
	\includegraphics[width=.8\textwidth]{问题三判断.png}
	\caption{ 指纹纹型判断示意图}
	\label{纹型判断}
\end{figure}

(1)若core点和delta点的数目为0,则认为是拱形。

(2)若core点和delta点的数目为2,则认为是螺旋形。

(3)若core点和delta点的数目为1,则可能是拱形或者是环形,于是利用以下方法进行进一步的确认。

连接core点和delta点,在拱形指纹中,直线方向与实际方向一致,而在环形指纹中,直线却穿过
指纹纹线。假设$\beta$为core点和delta点连线的方向角,$\alpha_1,\alpha_2,\alpha_3,...,\alpha_n$
为此线段上各段在指纹方向信息图上的方向角,如果平均值
         $ \frac{1}{n}\sum_{i=1}^n\sin(\alpha_i-\beta)$
小于经验值0.2,则认为该指纹为拱形,反之则为环形。

\subsection{指纹分类模型的求解}

利用matlab求解指纹图的core点和delta点个数后,机器自动识别了指纹的种类。通过查阅文献将自动识别与
人工比对进行对比并判断正确率,如下表:

% Table generated by Excel2LaTeX from sheet 'Sheet1'
\begin{table}[H]
	\centering
	\setlength{\abovecaptionskip}{5pt}
	\caption{识别对比}
	  \begin{tabular}{cccccc}
		\toprule[1.5pt]
		  & \multicolumn{2}{c}{算法识别结果} & \multicolumn{2}{c}{主观判断结果} & 准确率/\% \\
		  & 个数  & 占比/\% & 个数  & 占比/\% &  \\ \hline
	  环形  & 10  & 62.50\% & 11  & 68.75\% & 90.91\% \\
	  螺旋形 & 1   & 6.25\% & 3   & 18.75\% & 33.33\% \\
	  拱形  & 2   & 12.50\% & 2   & 12.50\% & 100\% \\
	  尖拱形 & 1   & 6.25\% & 0   & 0   & / \\
	  拒绝识别 & 2   & 12.50\% & 0   & 0   & / \\
	  \bottomrule[1.5pt]
	  \end{tabular}%
	\label{tab:addlabel}%
\end{table}%
  
综上,若以附件一中的指纹图像作为识别对象,该方法对于纹型的识别成功率可达81.25\%。



\clearpage

\section{模型总结与评价}


\subsection{模型优点}

1.在问题二中将指纹识别问题转化为了经典的指派问题,化繁为简。

2.在问题一中考虑到手指皮肤的弹性造成扭曲,因此在特征点提取时,选取了受影响的较小的中心部分,故结果可靠性较高。

3.在图像预处理中采用了结合方向信息的自适应动态阈值二值化方法,比传统的全局阈值二值化方法效果更好。

\subsection{模型缺点}
1.由于分析的指纹图像数量较少,因此在模型参数的选择上难免有局限性。

2.本文针对纹型分类的模型对于指纹图像的要求较为苛刻,若关键点被损坏则易发生拒认的情况。



\subsection{模型改进}

近年来,随着人工智能及相关技术的发展,学术界中提出了许多基于深度学习的指纹识别技术。在大量数据的加持下,指纹识别的准确率得到了极大的提升。

然而,深度学习算法的实施需要大量的数据。由于时间关系,未能额外搜寻到质量较高且被标记的学习样本。若将来时间允许,可以尝试开展相关的算法实施,进一步提升指纹识别的准确率

\clearpage

\addcontentsline{toc}{section}{参考文献}

%参考文献
\begin{thebibliography}{9}%宽度9
	\bibitem{bib:one} 郭玉兵. 指纹图像预处理算法研究[D]. 山东大学, 2011.
	\bibitem{bib:two} 李海燕, 程龙, 宗容,等. 基于三方向图的多尺度平滑指纹奇异点检测[J]. 华中科技大学学报(自然科学版), 2019(3).
	\bibitem{bib:three} 郭浩, 欧宗瑛. 基于Gabor滤波的指纹增强方法研究[J]. 仪器仪表学报, 2003, 24(0z2):384-386.
	\bibitem{bib:four}  楚亚蕴, 詹小四, 孙兆才,等. 一种结合方向信息的指纹图像二值化算法[J]. 中国图象图形学报, 2006, 11(006):855-860.
    \bibitem{bib:five}Guang-Min L , Xue-Jun C . Improvement of OPTA algorithm and its application in fingerprint images thinning[J]. Computer Engineering and Design, 2006.
    \bibitem{bib:five}张晓康. 基于全局信息的指纹配准及在匹配中的应用[D]. 2015.
	\bibitem{bib:five}Ravi J , , Raja K B , R, V K . Fingerprint Recognition Using Minutia Score Matching[J]. International Journal of Engineering Science Technology, 2009, 1(2).
	\bibitem{bib:five}陈晖, 殷建平, 祝恩,等. 一种基于细节点局部描述子的指纹图像匹配方法[J]. 计算机工程与科学, 2010, 32(1):87-91.
	\bibitem{bib:five}陶刚. 基于结构特征的指纹识别系统的匹配算法研究[D]. 华南师范大学, 2002.
	\bibitem{bib:five}Chen C H , Chen C Y , Hsu T M . Singular Points Detection in Fingerprints Based on Poincare Index and Local Binary Patterns[J]. Journal of Imaging ence and Technology, 2019, 63(3):030401.1-030401.7.

\end{thebibliography} 

\clearpage

\appendix %%附录
\section{结果}
指纹一:44.283 -1.510 0.748 1 4
46.872 -1.365 0.748 1 5
51.740 -1.236 0.805 1 6
19.026 -1.901 -0.459 1 0
13.928 -1.586 -0.286 1 1
28.792 -0.737 -0.435 1 2
65.521 -0.661 -0.616 1 6
71.868 0.818 -1.634 1 5
81.988 0.664 -2.331 1 6
34.132 -1.865 -0.358 2 0
96.773 -0.143 -0.692 2 12
39.051 -0.615 -0.499 2 3
51.225 0.513 0.760 2 4

指纹二:60.803 2.156 4.616 1 6
69.857 1.935 4.616 1 7
56.036 1.487 4.616 1 7
17.117 1.405 1.474 1 2
32.140 1.429 4.616 1 4
11.314 0.737 1.522 1 1
11.662 2.553 1.522 1 1
66.731 2.040 2.523 1 6
69.029 -0.019 4.613 1 6
166.018 0.327 1.932 2 10
70.937 2.123 1.474 2 2
140.071 1.423 2.195 2 1
112.969 1.201 4.616 2 4
23.022 1.767 2.496 2 9
118.152 1.840 2.499 2 7
40.200 2.317 2.606 2 7
41.110 2.456 2.709 2 8
98.955 2.764 2.731 2 8

指纹三:44.721 -1.731 -2.098 1 0
37.483 0.143 -0.848 1 3
28.018 -1.587 -0.544 1 0
31.016 -0.987 -0.553 1 0
79.101 -0.675 -0.848 1 5
11.180 -0.804 -0.694 1 0
64.382 -0.515 -0.693 1 2
14.318 0.514 -0.653 1 1
67.720 -0.908 -0.934 1 7
79.712 -0.970 -1.038 1 7
41.012 -1.409 -0.660 1 4
42.012 0.923 -0.480 1 4
72.422 -0.005 -0.438 1 8
53.600 0.307 -0.492 1 8
58.310 0.406 -0.492 1 9
94.069 -1.387 -0.668 1 11
90.907 0.255 -0.432 1 15
108.632 0.896 -1.396 2 2
39.051 -2.084 -0.721 2 1
27.166 0.161 -0.193 2 3

指纹四:81.633 0.177 2.060 1 6
90.802 0.273 2.272 1 8
47.927 0.310 1.548 1 5
47.539 0.462 2.075 1 6
19.416 0.115 1.006 1 2
42.190 0.622 1.411 1 4
40.311 0.309 0.452 1 4
98.489 0.943 2.405 1 11
88.119 0.233 -0.136 1 10
88.193 1.227 2.170 1 13
98.595 1.424 1.976 1 32
66.573 1.717 0.670 2 8
80.808 0.002 2.060 2 4
61.033 0.107 1.929 2 4
23.324 -0.313 0.780 2 3
124.310 1.003 0.520 2 7
149.967 0.107 0.628 2 1
53.160 1.677 0.677 2 1
12.207 0.968 1.438 2 4
12.207 0.209 -0.128 2 8

指纹五:76.381 0.398 -0.654 1 8
5.831 0.662 -0.369 1 1
9.849 -1.521 -0.189 1 1
66.068 0.513 0.158 1 6
82.680 0.641 0.307 1 9
34.986 0.662 -0.506 2 4
30.463 0.036 -0.553 2 3
194.258 -0.539 -0.017 2 11
58.856 -1.381 -0.189 2 1
133.060 -0.700 -0.757 2 6
119.017 -0.678 -0.909 2 7
18.868 -1.323 -0.211 2 2
67.676 0.692 -0.016 2 3
78.746 -0.030 -0.071 2 8
29.411 0.355 0.158 2 5
28.653 -1.097 -0.620 2 11
93.301 0.439 0.292 2 9

指纹六:76.381 0.398 -0.654 1 8
5.831 0.662 -0.369 1 1
9.849 -1.521 -0.189 1 1
66.068 0.513 0.158 1 6
82.680 0.641 0.307 1 9
34.986 0.662 -0.506 2 4
30.463 0.036 -0.553 2 3
194.258 -0.539 -0.017 2 11
58.856 -1.381 -0.189 2 1
133.060 -0.700 -0.757 2 6
119.017 -0.678 -0.909 2 7
18.868 -1.323 -0.211 2 2
67.676 0.692 -0.016 2 3
78.746 -0.030 -0.071 2 8
29.411 0.355 0.158 2 5
28.653 -1.097 -0.620 2 11
93.301 0.439 0.292 2 9


指纹七:95.603 -2.719 -4.295 1 11
94.372 -0.121 -1.355 1 8
94.578 -0.198 -1.355 1 8
81.320 -2.200 -1.358 1 8
59.506 -0.287 -1.347 1 3
36.346 -2.857 -4.003 1 3
34.928 -0.084 -1.398 1 2
21.095 0.052 -1.500 1 1
15.232 -0.258 -1.500 1 0
24.083 -1.341 -4.338 1 1
11.402 -1.158 -1.424 1 0
27.203 -2.053 -1.308 1 2
148.000 -0.530 -1.277 2 5
179.212 -2.120 -1.440 2 6
161.966 -0.657 -1.349 2 0
108.227 -2.580 -4.102 2 2
141.227 -1.054 -1.277 2 2
155.039 -1.244 -1.321 2 1
195.921 -1.341 -2.205 2 4
100.439 -1.690 -1.277 2 2
84.646 -0.258 -1.562 2 0
63.891 -1.109 -2.608 2 3
193.396 -1.728 -1.277 2 5
118.849 -0.984 -2.650 2 3
38.897 -2.183 -1.277 2 6
27.313 0.013 -1.739 2 3
52.555 -0.520 -2.630 2 4
22.361 -0.734 -2.764 2 5
120.037 -0.180 -1.918 2 5


指纹八:78.262 -0.133 -3.702 1 7
57.245 -0.008 -3.840 1 5
55.902 -0.133 -3.795 1 5
51.313 -0.247 -3.795 1 5
24.331 0.165 -1.003 1 0
28.302 -2.253 -1.033 1 2
32.202 -0.791 -3.810 1 4
68.964 -0.946 -2.058 1 7
55.154 -0.850 -1.900 1 5
62.626 -0.898 -2.058 1 6
93.392 -1.014 -2.424 1 9
54.562 -0.520 -2.049 1 5
60.141 -0.420 -2.327 1 6
152.411 -2.515 -3.851 2 8
129.016 0.285 -3.811 2 4
139.176 -2.134 -4.041 2 7
75.286 -2.474 -0.914 2 3
172.540 -1.122 -2.058 2 6
66.068 -2.623 -2.187 2 5
71.840 -0.133 -2.204 2 5


指纹九:56.613 -1.481 -0.516 1 16
58.822 -1.149 -0.460 1 17
81.216 -0.360 -1.538 1 11
28.636 -0.965 -0.379 1 3
7.071 0.252 -0.533 1 1
28.018 -0.569 -0.327 1 2
20.100 -0.433 -0.327 1 1
57.315 -0.428 -0.389 1 4
24.166 -0.960 -0.782 1 2
80.324 -0.203 -0.528 1 6
65.947 0.177 -0.227 1 12
71.063 0.996 -0.215 1 4
74.673 0.903 -0.215 1 4
63.063 -1.011 -0.589 2 14
38.897 -1.300 -0.379 2 9
32.573 -0.845 -1.175 2 3
67.417 -0.897 -1.775 2 5
103.392 0.444 -0.497 2 14


指纹十:58.523 2.196 0.425 1 5
57.706 2.058 0.425 1 5
9.220 2.327 0.974 1 0
10.296 -0.089 1.330 1 0
63.159 1.381 1.594 1 3
85.563 0.181 3.520 1 10
164.454 -0.503 1.055 2 6
110.386 1.804 3.520 2 10
105.380 1.749 3.520 2 6
75.326 -0.242 1.387 2 4
96.260 0.463 1.064 2 9
112.058 1.277 3.520 2 7
67.179 2.274 0.523 2 1
49.041 1.438 0.423 2 2
94.021 1.132 3.520 2 7
80.654 0.833 1.351 2 1
18.682 0.266 0.467 2 1
29.069 1.170 1.717 2 5
69.871 0.445 3.520 2 6
35.355 2.036 1.491 2 3
106.170 1.634 1.761 2 4
18.439 1.795 1.840 2 5

指纹十一:59.464 -0.251 1.314 1 6
27.514 1.310 0.850 1 2
23.022 1.020 0.783 1 2
2.236 -0.130 0.977 1 0
32.388 0.822 1.100 1 4
15.000 0.333 0.783 1 2
57.585 0.622 0.761 1 5
82.462 1.222 0.777 1 6
64.078 1.378 0.818 1 5
75.743 1.809 1.141 1 7
69.921 2.258 1.187 1 7
98.671 -0.285 1.210 1 10
73.430 1.489 0.430 2 6
79.831 1.191 0.383 2 6
33.734 1.186 0.850 2 2
99.464 1.200 0.615 2 7
62.642 -0.302 1.223 2 6
82.420 1.926 1.235 2 7

指纹十二:71.169 -2.842 -1.571 1 4
14.036 -1.642 -1.571 1 1
36.056 -1.626 -1.571 1 3
18.439 -2.279 -1.571 1 1
32.757 -2.116 -1.571 1 2
123.483 -2.248 -1.571 2 3
138.293 -1.636 -1.571 2 2
78.243 -1.435 -1.571 2 2
46.098 -0.636 -1.571 2 4
73.682 -2.921 -1.571 2 6


指纹十三:74.277 0.228 1.562 1 5
72.367 1.236 0.578 1 6
41.110 0.720 1.096 1 1
4.472 0.578 1.067 1 0
26.401 0.391 1.196 1 3
19.925 -0.223 1.111 1 2
69.584 -0.399 1.148 2 4
6.000 1.042 1.042 2 0
62.032 1.074 1.032 2 2
47.170 0.957 1.155 2 4
103.121 0.490 1.143 2 1
15.264 0.454 1.304 2 3
36.056 0.544 1.257 2 4
52.355 2.240 1.097 2 4


指纹十四:75.286 2.450 4.712 1 9
62.610 2.678 4.712 1 7
50.596 2.820 4.712 1 6
54.489 0.746 1.866 1 4
93.434 1.474 4.422 1 3
34.928 1.983 4.638 1 3
44.385 1.964 4.638 1 3
58.822 1.260 4.712 1 7
68.964 1.865 4.488 1 4
50.537 2.043 4.638 1 2
60.415 2.071 4.547 1 5
90.554 2.609 4.700 1 9
226.080 0.686 4.712 2 3
28.425 2.290 1.653 2 6
138.101 2.815 4.712 2 6
85.041 2.774 4.712 2 9


指纹十五:68.797 -0.936 0.338 1 1
64.846 0.364 -0.540 1 6
5.000 0.067 0.371 1 0
35.355 -0.075 -0.360 1 5
92.195 0.001 -0.540 1 12
9.220 0.929 0.371 1 0
39.825 -0.612 -0.348 1 5
51.313 -0.927 0.024 1 6
111.987 -1.282 0.149 2 4
86.122 0.000 -0.074 2 5
78.000 -1.332 -0.007 2 3
117.924 0.434 0.067 2 1
70.036 -0.080 0.730 2 1
143.753 0.264 1.344 2 2
13.928 -0.494 -0.430 2 5
34.366 1.456 -1.427 2 4
228.055 -0.989 -0.554 2 7
50.990 -1.101 -0.716 2 9


指纹十六:18.974 -0.360 0.880 1 2
89.140 -0.626 1.227 1 10
98.509 -0.580 1.227 1 11
126.194 -0.074 0.960 2 8
96.260 -0.284 0.780 2 5
82.462 -0.134 0.882 2 4
84.380 1.660 0.265 2 10
97.591 0.141 0.932 2 7
94.112 1.269 1.038 2 4
142.678 0.663 0.871 2 4
48.466 1.167 0.977 2 3
62.586 0.834 0.634 2 6
36.401 1.237 0.876 2 1
91.137 0.793 0.253 2 9
11.705 -0.012 1.008 2 3
83.385 1.468 0.774 2 5
128.316 -0.496 1.282 2 10




\section{代码}
\subsection{Gabor matlab 源程序}
\begin{lstlisting}[language=matlab]
	%------------------------------------------------------------------------
	%gabor_kernel
	%creates a 2D gabor convolution mask
	%------------------------------------------------------------------------
	function [gr,gi] = gabor_kernel(dx,dy,f,theta)
		dx    = round(dx);
		dy    = round(dy);
		[x,y] = meshgrid(-3*dx:3*dx,-3*dy:3*dy);
		xp    = x*cos(theta)+y*sin(theta);
		yp    = -x*sin(theta)+y*cos(theta);
		gr     = exp(-xp.^2/dx.^2-yp.^2/dy.^2).*cos(2*pi/f*xp);
		gi     = exp(-xp.^2/dx.^2-yp.^2/dy.^2).*sin(2*pi/f*xp);
	%end function gabor_kernel

	
\end{lstlisting}

\subsection{problem1.m matlab 源程序}
\begin{lstlisting}[language=matlab]
	function yi=prepro_1(img,msk);
N=16;
msk=pad_image(msk,N);
img  =  double(img);
y_img=img;
img = pad_image(img,N);
[ht,wt]=size(img);
    nimg    =   normalize_image(img,0,100);
    %---------------------------------------
    %orientation image
    %---------------------------------------
    oimg            =   blk_orientation_image(img,N);
    %---------------------------------------
    %smoothen orientation image
    %---------------------------------------
    oimg            =   smoothen_orientation_field(oimg);
%求指纹频率场
    [x,y]           =   meshgrid(-8:7,-16:15);
    [blkht,blkwt]   =   size(oimg);
    [ht,wt]         =   size(img);
    yidx = 1; %index of the row block
    for i = 0:blkht-1
        row     = (i*N+N/2);%+N for the pad
        xidx    = 1; %index of the col block
        for j = 0:blkwt-1   
            col = (j*N+N/2);
            %row,col indicate the index of the center pixel
            th  = oimg(yidx,xidx);
            u = x*cos(th)-y*sin(th);
            v = x*sin(th)+y*cos(th);
            u           =   round(u+col); u(u<1)  = 1; u(u>wt) = wt;
            v           =   round(v+row); v(v<1)  = 1; v(v>ht) = ht;
            %find oriented block
            idx         =   sub2ind(size(img),v,u);
            blk         =   img(idx);
            blk         =   reshape(blk,[32,16]);
            %find x signature
            xsig        =   sum(blk,2);
            f(yidx,xidx) = find_peak_distance(xsig);
            xidx = xidx +1;
        end;
        yidx = yidx +1;
    end;
    fimg=filter_frequency_image(f);
    y = do_gabor_filtering(img,oimg,fimg);     %gobor滤波增强图像
   yi=imscale(y);

    

	
\end{lstlisting}

\subsection{problem2.m 源程序}
\begin{lstlisting}[language=matlab]
	clear, clc;
[endpoints5, bifurcations5] = get_code(10);
[endpoints6, bifurcations6] = get_code(11);
for i = 1:size(endpoints5,1)
    feature1(i,:)= [endpoints5(i,7), endpoints5(i,5), endpoints5(i,6), 1, endpoints5(i,3)];
end
for i = size(endpoints5,1)+1:size(endpoints5,1)+size(bifurcations5,1)
    feature1(i,:)= [bifurcations5(i-size(endpoints5,1),7), bifurcations5(i-size(endpoints5,1),5), bifurcations5(i-size(endpoints5,1),6), 2, bifurcations5(i-size(endpoints5,1),3)];
end

for i = 1:size(endpoints6,1)
    feature2(i,:)= [endpoints6(i,7), endpoints6(i,5), endpoints6(i,6), 1, endpoints6(i,3)];
end
for i = size(endpoints6,1)+1:size(endpoints6,1)+size(bifurcations6,1)
    feature2(i,:)= [bifurcations6(i-size(endpoints6,1),7), bifurcations6(i-size(endpoints6,1),5), bifurcations6(i-size(endpoints6,1),6), 2, bifurcations6(i-size(endpoints6,1),3)];
end
for i = 1:size(feature1,2)
    feature1(:,i) = (feature1(:,i)-min(feature1(:,i)))/(max(feature1(:,i))-min(feature1(:,i)));
    feature2(:,i) = (feature2(:,i)-min(feature2(:,i)))/(max(feature2(:,i))-min(feature2(:,i)));
end
d = zeros(size(feature1,1),size(feature2,1));
for i = 1:size(feature1,1)
    for j = 1:size(feature2,1)
        d(i,j)=sqrt((feature1(i,1)-feature2(j,1))^2+(feature1(i,2)-feature2(j,2))^2+(feature1(i,3)-feature2(j,3))^2+(feature1(i,4)-feature2(j,4))^2+(feature1(i,5)-feature2(j,5))^2);
    end
end
fid = fopen('d.txt','w');
for i = 1:size(d,1)
    for j = 1:size(d,2)
        fprintf(fid, '%.4f ', d(i,j));
    end
    fprintf(fid, '\n');
end
fclose(fid);

	
\end{lstlisting}

\subsection{smoothen orientation field--matlab 源程序}
\begin{lstlisting}[language=matlab]
	function oimg = smoothen_orientation_field(oimg)
    g   =   cos(2*oimg)+i*sin(2*oimg);
    g   =   imfilter(g,fspecial('gaussian',5));
    oimg=   0.5*angle(g);
    save('oimg1.mat','oimg');
%end function smoothen_orientation_field
%blk_frequency_image
\end{lstlisting}


\subsection{get code--matlab 源程序}
\begin{lstlisting}[language=matlab]
	function [endpoints, bifurcations]=get_code(ord)
	[endpoint_x, endpoint_y,bifurcation_x,bifurcation_y, endpoint_line_num, bifurcation_line_num, corepoint_y, corepoint_x,oimg] = solve_pro(ord);
	%% 筛选特征点
	R = 100;
	% 对端点进行筛选
	endpoints = []; % 第一列为x坐标,第二列为y坐标,第三列为穿过脊的数量,第四列为方向场,第五列为极坐标系下的角度
					% 第六列为方向场角度差值, 第7列为距离,第8列为ρ
	bifurcations = [];
	for i = 1:length(endpoint_x)
		if sqrt((corepoint_x-endpoint_x(i))^2+(corepoint_y-endpoint_y(i))^2) <= R
			endpoints(end+1,1)=endpoint_x(i);
			endpoints(end,2) = endpoint_y(i);
			endpoints(end,3) = endpoint_line_num(i);
		end
	end
	for i = 1:length(bifurcation_x)
		if sqrt((corepoint_x-bifurcation_x(i))^2+(corepoint_y-bifurcation_y(i))^2) <= R
			bifurcations(end+1,1)=bifurcation_x(i);
			bifurcations(end,2) = bifurcation_y(i);
			bifurcations(end,3) = bifurcation_line_num(i);
		end
	end
	
	%% 计算方向场角度差
	% 中心点所在角度场
	core_angle = oimg(floor(corepoint_y/16), floor(corepoint_x/16));
	% 计算端点
	for i = 1:size(endpoints,1)
		endpoints(i,4) = oimg(floor(endpoints(i,2)/16),floor(endpoints(i,1)/16));
		endpoints(i,4) = endpoints(i,4)-core_angle;
		if endpoints(i,1) == corepoint_x
			endpoints(i,5) = 0;
		else
			endpoints(i,5) = atan((endpoints(i,2)-corepoint_y)/(endpoints(i,1)-corepoint_x))-core_angle;
		end
		endpoints(i,6) = endpoints(i,4)-core_angle;
		endpoints(i,7) = sqrt((corepoint_x-endpoints(i,1))^2+(corepoint_y-endpoints(i,2))^2);
	end
	%计算交叉点
	for i = 1:size(bifurcations,1)
		bifurcations(i,4) = oimg(floor(bifurcations(i,2)/16),floor(bifurcations(i,1)/16));
		bifurcations(i,4) = bifurcations(i,4)-core_angle;
		if bifurcations(i,1) == corepoint_x
			bifurcations(i,5) = 0;
		else
			bifurcations(i,5) = atan((bifurcations(i,2)-corepoint_y)/(bifurcations(i,1)-corepoint_x))-core_angle;
		end
		bifurcations(i,6) = bifurcations(i,4)-core_angle;
		bifurcations(i,7) = sqrt((corepoint_x-bifurcation_x(i))^2+(corepoint_y-bifurcation_y(i))^2);
	end
\end{lstlisting}


\subsection{get picture--matlab 源程序}
\begin{lstlisting}[language=matlab]
	clear, clc;
	for i = 7:16
	[img] = solve_pro(i);
	figure,imshow(img);
	saveas(gcf, [num2str(i),'ԭͼ'], 'png')
	end
\end{lstlisting}


\subsection{问题二模型求解--lingo 源程序}
\begin{lstlisting}[language=lingo]
	model:
	sets:
	row1/1..22/;
	row2/1..18/;
	col/1..5/;
	dist(row1,row2):d;
	mat(row1, row2):x,z,p;
	endsets
	data:
	d = 1.3979 0.6704 0.6900 1.2423 0.9313 0.8982 0.7611 0.7579 0.6387 0.9542 0.9641 1.5458 1.1176 1.1739 1.2166 1.3047 1.6811 1.4630 
	1.3668 0.6478 0.6594 1.2065 0.9052 0.8593 0.7255 0.7397 0.6202 0.9513 0.9697 1.5187 1.1095 1.1616 1.2025 1.2946 1.6549 1.4629 
	1.5304 0.5911 0.6186 1.0399 0.9599 0.8280 1.0170 1.1245 0.9255 1.2235 1.2018 1.8640 1.4151 1.4839 1.1938 1.6327 1.8087 1.6524 
	1.1020 0.6177 0.4886 0.3541 0.7522 0.2909 0.7907 1.1172 0.9677 1.3354 1.4210 1.5310 1.5005 1.5043 1.1720 1.6347 1.4598 1.7523 
	0.9757 0.1851 0.2298 0.7979 0.4654 0.4836 0.4277 0.5702 0.3624 0.7363 0.7928 1.2649 1.1605 1.2012 1.0164 1.2683 1.3725 1.3107 
	0.4674 1.0452 1.0573 1.1846 0.6962 1.0465 0.7892 0.8539 0.8522 0.7309 0.8517 0.5637 1.5263 1.5397 1.4254 1.4251 1.1156 1.2662 
	1.3428 1.4809 1.4478 1.5977 1.4246 1.4238 1.1691 1.1971 1.2816 1.4567 1.5928 1.2760 0.7650 0.6503 1.0216 0.5970 0.7948 1.1421 
	1.3403 1.4408 1.4987 1.7650 1.2926 1.5992 1.3474 1.2536 1.2486 1.0656 1.0699 1.3349 1.0401 1.1114 1.0248 0.9094 0.9138 0.3608 
	1.2661 1.2510 1.3144 1.5511 1.1471 1.4224 1.2467 1.1898 1.1460 1.0315 1.0374 1.3907 0.9612 1.0378 0.7359 0.8730 0.8019 0.2547 
	1.2435 1.1803 1.1310 1.2036 1.1596 1.0795 1.0545 1.2104 1.1819 1.4102 1.5140 1.4215 0.7558 0.7189 0.5710 0.8174 0.6609 1.1044 
	1.3501 1.3231 1.3003 1.5444 1.2805 1.3163 1.0955 1.1274 1.1544 1.2962 1.3866 1.3311 0.5219 0.4978 0.8309 0.5431 0.8333 0.9439 
	1.1773 1.2885 1.3375 1.5413 1.1414 1.4150 1.2136 1.1707 1.1511 1.0400 1.0759 1.2641 0.9601 1.0151 0.7842 0.8242 0.6455 0.3030 
	1.7768 1.1692 1.1872 1.5336 1.3902 1.3282 1.3126 1.3250 1.2345 1.4636 1.4680 1.9685 0.6771 0.7680 0.6269 0.9775 1.4257 1.1643 
	1.6473 1.1203 1.1024 1.3768 1.3028 1.1798 1.2076 1.2924 1.2056 1.4770 1.5124 1.8523 0.6237 0.6810 0.5171 0.9353 1.2565 1.2019 
	1.1502 1.2581 1.2998 1.4766 1.1035 1.3623 1.2000 1.1921 1.1566 1.0674 1.1042 1.2757 0.9805 1.0361 0.7400 0.8772 0.5993 0.4126 
	1.4003 1.0563 1.0438 1.2247 1.1554 1.0886 1.0934 1.1889 1.1194 1.3552 1.4184 1.6295 0.6620 0.6830 0.2852 0.8168 0.9287 1.0151 
	1.6012 1.1899 1.1253 1.2020 1.3237 1.0901 1.2570 1.4512 1.3580 1.6700 1.7364 1.8738 0.9409 0.9438 0.6426 1.1817 1.1949 1.4539 
	1.3738 1.0564 1.0503 1.2607 1.0911 1.0990 1.1171 1.2232 1.1247 1.2830 1.3147 1.5998 0.7226 0.7962 0.3682 0.9105 0.9155 0.9225 
	1.0670 1.2297 1.2447 1.3098 1.0536 1.2490 1.1811 1.2607 1.2070 1.1907 1.2506 1.2813 1.0731 1.1080 0.6879 1.0040 0.4183 0.7015 
	1.5702 1.0557 1.0790 1.3760 1.1921 1.1999 1.2215 1.2723 1.1541 1.3156 1.3107 1.8046 0.7306 0.8390 0.3984 0.9784 1.1882 0.9576 
	1.3800 1.0925 1.1250 1.4373 1.1460 1.2411 1.0813 1.0516 1.0105 1.1216 1.1580 1.5044 0.4477 0.5351 0.4068 0.5411 0.9123 0.6138 
	1.4871 1.0795 1.0963 1.3568 1.1444 1.1890 1.2108 1.2824 1.1649 1.2855 1.2827 1.7186 0.8008 0.9041 0.4639 1.0094 1.0879 0.9157 ;
	enddata
	max = M/22;
	L = 1.0;
	@for(row2(j):@sum(row1(i):x(i,j))<1);
	@for(row1(i):@sum(row2(j):x(i,j))<1);
	@for(row1(i):@for(row2(j):(1-z(i,j))>x(i,j)*d(i,j)-L));
	@for(row1(i):@for(row2(j):p(i,j)=x(i,j)*z(i,j)));
	@for(row1(i):@for(row2(j):p(i,j)<x(i,j)));
	@for(row1(i):@for(row2(j):p(i,j)<z(i,j)));
	@for(row1(i):@for(row2(j):x(i,j)+z(i,j)-1<p(i,j)));
	M = @sum(row1(i):@sum(row2(j):z(i,j)*x(i,j)));
	@for(mat:@bin(x));
	@for(mat:@bin(z));
\end{lstlisting}

\subsection{get main--matlab 源程序}
\begin{lstlisting}[language=matlab]
	clear,clc;
	[endpoints, bifurcations] = get_code(1);
	fid = fopen('result.txt','wb');
	for i = 1:size(endpoints,1)
		fprintf(fid, '%.3f %.3f %.3f %d %d\n', endpoints(i,7), endpoints(i,5), endpoints(i,6), 1, endpoints(i,3));
	end
	for i = 1:size(bifurcations,1)
		fprintf(fid, '%.3f %.3f %.3f %d %d\n', bifurcations(i,7), bifurcations(i,5), bifurcations(i,6), 2, bifurcations(i,3));
	end
	fclose(fid);isting}
\end{lstlisting}

\subsection{细化--matlab 源程序}
\begin{lstlisting}[language=matlab]
	function c=thin_img(iseg, msk)

c=bwmorph(iseg,'thin','inf');   %
fun=@minutie;
c = nlfilter(c,[3 3],fun);

i_thin=c(5:364,1:256);
msk=msk(5:364,1:256);  % ó
i_thin(i_thin>0)=1;

% imshow(c);
% title('细 化');
\end{lstlisting}
\end{spacing}
\end{document}